\section{Development}

When trying to follow a traditional software engineering approach to in
Haskell, one soon runs into several dead ends: due to the different paradigm
and style, trying to apply some methods feels forced or unnatural.
Traditionally, in Haskell the approach when formalizing an approach just
involves a mathematical approach, using just Mathematics for a formal design
through equations, that are then verified by a theorem prover such as Agda or
Coq. Furthermore, the problems only increase when we try to use UML: this
methodology was clearly not designed for other than Object-Oriented programming
and it is not possible to create traditional diagrams for Haskell without
running in unforgivable simplifications, inaccuracies regarding the system or
simply nonsensical diagrams. \\

However, due to the academic character of this thesis, it is needed to provide
some formal specification of the system using the learned methodologies for
software development, so we will try to use the appropriate mathematical
concepts embedded in traditional formal requirements, function specifications,
etc; as well as more familiar specification systems for Haskell projects like
the type specification. Please, bear in mind that this is not usually the case
in Haskell projects and the recommended guidelines just include formal
specification through mathematical definition. \\

\subsection{Use Cases}

To start designing the library, first we have to formally specify the use cases
of possible users of the library. The agents involved in this specifications
are simply the end user and the library itself, which will provide functions
for the user by demand.\\

\begin{table}[h]
  \centering
  \rowcolors{2}{gray90}{white}
\begin{tabular}{| p{4cm} | p{6cm} |}
  \hline
  \texttt{ID}            & \texttt{UC-XX} \\
  \hline
  \textit{Title}         & \\
  \textit{Actor}         & \\
  \textit{Preconditions} & \\
  \textit{Description}   & \\
  \hline
\end{tabular}
\caption{Use Case template}
\label{table:uc-ex}
\end{table}

All the use cases that will be covered will be specified in individual tables
following the template in the Table \ref{table:uc-ex}. The complete set of use
cases for the library is stated following this paragraph.

\subsection{Functional Requirements}
\subsection{Implementation}

\newpage