\section{Development}

When trying to follow a traditional software engineering approach to in
Haskell, one soon runs into several dead ends: due to the different paradigm
and style, trying to apply some methods feels forced or unnatural.
Traditionally, in Haskell the approach when formalizing an approach just
involves a mathematical approach, using just Mathematics for a formal design
through equations, that are then verified by a theorem prover such as Agda or
Coq. Furthermore, the problems only increase when we try to use UML: this
methodology was clearly not designed for other than Object-Oriented programming
and it is not possible to create traditional diagrams for Haskell without
running in unforgivable simplifications, inaccuracies regarding the system or
simply nonsensical diagrams. \\

However, due to the academic character of this thesis, it is needed to provide
some formal specification of the system using the learned methodologies for
software development, so we will try to use the appropriate mathematical
concepts embedded (as long as they are manageable and in the scope of the
thesis), as well as more familiar specification systems for Haskell projects
like the type specification. Please, bear in mind that this is not usually the
case in Haskell projects and the recommended guidelines just include formal
specification through mathematical definition and theorem proving. \\

\subsection{Use Cases}

To start designing the library, first we have to formally specify the use cases
of possible users of the library. The agents involved in this specifications
are simply the end user and the library itself, which will provide functions
for the user by demand.\\

\begin{table}[h]
  \centering
  \rowcolors{2}{gray90}{white}

\begin{tabular}{
  !{\color{azulUC3M}\vline} p{4cm}
  !{\color{azulUC3M}\vline} p{6cm}
  !{\color{azulUC3M}\vline}}
  \arrayrulecolor{azulUC3M}
  \rowcolor{azulUC3M}

  \multicolumn{1}{l !{\color{white}\vline}}
  {\color{white}{\texttt{ID}}}
  & \multicolumn{1}{l}
    {\color{white}{\texttt{UC-XX}}} \\
  
  \textit{Title}         & \\
  \textit{Actor}         & \\
  \textit{Preconditions} & \\
  \textit{Description}   & \\
  \hline
\end{tabular}
\caption{Use Case template}
\label{table:uc-ex}
\end{table}

All the use cases that will be covered will be specified in individual tables
following the template in the Table \ref{table:uc-ex}. Each of the use cases
will receive a unique identifier of the format \texttt{UC-XX}, where
\texttt{XX} is a double digit number. This unique identifier will be used later
on different matrices to trace requirements. The complete set of use cases for
the library is stated following this page.

\newpage

\begin{uc3m-table}{UC-01}{Use Case \texttt{UC-01}}
  \textit{Title} & \textbf{Solve a search problem} \\
  \textit{Actor} & User \\
  \textit{Preconditions} &
  The user has a compatible representation of the problem programmed in
  Haskell, and the package \texttt{agis} is already installed in their system.
  \\
  \textit{Description} &
  The user imports the module containing the algorithm that he wants to use,
  and includes the functionality in their code. Then, the user can run the code
  to obtain the solution found.\\
\end{uc3m-table}


\begin{uc3m-table}{UC-02}{Use Case \texttt{UC-02}}
  \textit{Title} & \textbf{Solve a search problem and get statistics} \\
  \textit{Actor} & User \\
  \textit{Preconditions} &
  The user has a compatible representation of the problem programmed in
  Haskell, and the package \texttt{agis} is already installed in their system.
  \\
  \textit{Description} &
  The user imports the module containing the monadic version of the algorithm
  that he wants to use, and includes the functionality in their code. Then, the
  user can run the code to obtain the solution found and several search
  statistics.\\  
\end{uc3m-table}


\begin{uc3m-table}{UC-03}{Use Case \texttt{UC-03}}
  \textit{Title} & \textbf{Design a new search algorithm} \\
  \textit{Actor} & User \\
  \textit{Preconditions} &
  The package \texttt{agis} is already installed in the user's system. \\
  \textit{Description} &
  The user can import the module containing several functions that they can use
  to build their algorithm, as well as using a monadic version of those
  functions to gain a better understanding on the algorithm.\\ 
\end{uc3m-table}


\begin{uc3m-table}{UC-04}{Use Case \texttt{UC-04}}
  \textit{Title} & \textbf{Test a new algorithm} \\
  \textit{Actor} & User \\
  \textit{Preconditions} &
  The package \texttt{agis} is already installed in the user's system, and the
  user has already implemented their algorithm using the library types and
  functions. \\
  \textit{Description} &
  The user can import several toy problems that the library offers to test that
  algorithm to check the behavior or performance.\\ 
\end{uc3m-table}

\begin{uc3m-table}{UC-05}{Use Case \texttt{UC-05}}
  \textit{Title} & \textbf{Compare a new algorithm} \\
  \textit{Actor} & User \\
  \textit{Preconditions} &
  The package \texttt{agis} is already installed in the user's system, and the
  user has already implemented their algorithm using the library types and
  functions.\\
  \textit{Description} &
  The user can import more algorithms from the library and trivially apply one
  or another to the same problem space, to check the performance of both of
  them side by side.\\
\end{uc3m-table}



\newpage 

\subsection{Functional Requirements}
\subsection{Implementation}

\newpage