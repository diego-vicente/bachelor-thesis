% English summary
\vspace*{\fill}

\selectlanguage{english}
\begin{abstract}

There are currently few experiments that try to perform heuristic search in
Haskell, even though the language provides a clean and modular syntax
that can be specially useful for educational or research purposes in the field.
Furthermore, functional programming provides a inherently concurrent way of
writing code, which results in trivial parallelism using compiler flags. Also,
due to the particular constraints imposed by the language, it is specially
interesting to create a library of algorithms that solves the $k$ best paths
problem. Taking into account that it can be detrimental to the overall
performance, the language's nature allows the framework to return a list (that
can theoretically be as long as infinite) containing all the solutions in the
problem space, and use its lazy evaluation to only compute the $k$ solutions
asked by the user, without any kind of code modification.\\

The aforementioned reasons are found as a motivation for developing a framework
in Haskell, that allows the user to perform searches, as well as designing
different algorithms and test them along with the ones provided for comparison.
Haskell's strong type system let us declare a set of types used by the library,
declaring all needed constraints in them and ensuring that if the code provided
by the user type checks the code will most likely behave as expected by the
user.\\

This thesis details all the steps that were needed in the process: from a
software engineering approach to the design, to the generation of all needed
artifacts to ensure the quality and intuition about the library; including the
implementation step and all the reasoning behind it, and all formal
verification of the models used in the library. The testing process also
contains all the performance checks along with several low-level profile tests
to spot a space leak in the library. All the organizational, legal and
socioeconomic details of the project are also included in the document, along
with an appendix containing all the generated documentation of the framework.\\

The result is \texttt{agis}: a Haskell package that lets the user include in
their own code a set of types to model and solve heuristic problems; design
modular heuristic search algorithms (using intermediate functions provided by
the library) or write them from scratch (using the types provided by the
library) and be able to run them in several search domains provided in the
framework or run several benchmarks using a \texttt{criterion} interface.\\
\end{abstract}

\vspace*{\fill}

\newpage

% Spanish summary
\vspace*{\fill}

\selectlanguage{spanish}
\begin{abstract}

Actualmente existen pocos ejemplos de búsqueda heurística en Haskell, a pesar
de que el lenguaje ofrece una sintaxis modular y limpiar que lo hace
especialmente interesante para fines educativos o de investigación en este
campo. Además, la programación funcional ofrece una forma de escribir código
intrínsecamente concurrente, lo que hace código de Haskell trivialmente
paralelizable (tan solo es necesario activar ciertas opciones propuestas por el
compilador). Además de todo ello, las restricciones impuestas por el lenguaje
hacen una opción especialmente interesante que los algoritmos incluídos en la
librería no tengan un comportamiento clásico, sino que resuelvan a su vez el
problema de las $k$ mejores soluciones: el algoritmo puede devolver una lista
con todas las soluciones incluídas en el espacio de búsqueda (que puede ser
infinita), y usar la evaluación perezosa del lenguaje para solo calcular las
que el usuario realmente pida, sin ninguna modificación de código ni problema a
la hora de evaluar dicha lista.\\

Todas estas razones sirven de motivación para el desarrollo de un framework en
Haskell, que permita al usuario tanto ejecutar búsquedas como diseñar distintos
algoritmos y compararlos con otros ya existentes. El sistema de tipos de
Haskell permite a la librería declarar un conjunto de tipos para que sean
usados por el usuario a la hora de modelar problemas o diseñar algoritmos y
asegurar de esta manera que si el código compila, es muy probable que el
comportamiento del código sea el esperado.\\

En este trabajo se recogen todos los pasos que se han seguido en este proceso:
desde un diseño a nivel de ingeniería de software hasta la generación de todos
los artefactos necesarios para asegurar un correcto diseño de la librería, para
pasar a la implementación de dicho diseño. Esta implementación incluye todas
las justificaciones necesarias así como la verificación formal de los modelos
que así lo requiren. Se incluye una sección que define todas las pruebas
realizadas que incluyen la medición de tiempos y recursos, así como varias
pruebas de bajo nivel realizadas para encontrar el origen de una pérdida de
espacio y desempeño encontrada en la librería. Todos los los detalles
organizativos, legales y socioeconómicos se encuentran recogidos en este
documento, así como la documentación generada por la librería en un apéndice a
este.\\

El resultado del trabajo es \texttt{agis}: un paquete de Haskell que permite al
usuario incluir en su propio código un conjunto de tipos para modelar problemas
y resolverlos usando distintos algoritmos de búsqueda, diseñar algoritmos de
búsqueda modulares (usando funciones intermedias proporcionadas por el
framework) o implementar uno de cero (usando los tipos de la librería), y ser
así capaz de correrlos en distintos dominios de búsqueda incluidos en el
framework o correr distintos benchmarks con la interfaz proporcionada al
paquete \texttt{criterion}.\\
\end{abstract}

\vspace*{\fill}
\newpage


\selectlanguage{english}

\vspace*{4cm}

\section*{\centering{Acknowledgements}}

Thanks to Carlos Linares López, the supervisor of this thesis, for his
guidance. Without his trust in my own judgment and work, this thesis would
have never become what it came to be.\\

\selectlanguage{spanish}

Gracias a mis padres, por su apoyo y dedicación a lo largo de toda mi
educación. Este trabajo es tan vuestro como mío, aunque sé que siempre
pensásteis que me quedaría ciego de tanto mirar la pantalla antes de
terminar de escribirlo.\\

\vspace{1cm}

\selectlanguage{english}

\centerline{\emph{With a little help of my friends.}}

\newpage

%%% Local Variables:
%%% TeX-master: "tfg"
%%% End: