% English summary
\vspace*{\fill}

\selectlanguage{english}
\begin{abstract}

There are currently few experiments that try to perform heuristic search in
Haskell, even though the language provides a clean and modular syntax
that can be specially useful for educational or research purposes in the field.
Furthermore, functional programming provides a inherently concurrent way of
writing code, which results in trivial parallelism using compiler flags. Also,
due to the particular constraints imposed by the language, it is specially
interesting to create a library of algorithms that solves the $k$ best paths
problem. Taking into account that it can be detrimental to the overall
performance, the language's nature allows the framework to return a list (that
can theoretically be as long as infinite) containing all the solutions in the
problem space, and use its lazy evaluation to only compute the $k$ solutions
asked by the user, without any kind of code modification.\\

The aforementioned reasons are found as a motivation for developing a framework
in Haskell, that allows the user to perform searches, as well as designing
different algorithms and test them along with the ones provided for comparison.
Haskell's strong type system let us declare a set of types used by the library,
declaring all needed constraints in them and ensuring that if the code provided
by the user type checks the code will most likely behave as expected by the
user.\\

This thesis details all the steps that were needed in the process: from a
software engineering approach to the design, to the generation of all needed
artifacts to ensure the quality and intuition about the library, to the
implementation step and all the reasoning behind it; including all formal
verification of the models used in the library. The testing process also
contains all the performance checks along with several low-level profile tests
to spot a space leak in the library. All the organizational, legal and
socioeconomic details of the project are also included in the document, along
with an appendix containing all the generated documentation of the framework.\\

The thesis result is \texttt{agis}: a Haskell package that lets the user
include in their own code a set of types to model and solve heuristic problems;
design modular heuristic search algorithms (using intermediate functions
provided by the library) or write them from scratch (using the types provided
by the library) and be able to run them in several search domains provided in
the framework or run several benchmarks using a \texttt{criterion} interface.\\
\end{abstract}

\vspace*{\fill}

\newpage

% Spanish summary
\selectlanguage{spanish}
\begin{abstract}

\end{abstract}

\newpage

\selectlanguage{english}

\section*{Acknowledgement}

\newpage

%%% Local Variables:
%%% TeX-master: "tfg"
%%% End: