\section{Objectives}

\subsection{Parts of the Library}

The main goal of this project is to develop a complete framework for Haskell,
containing the following parts:

\begin{enumerate}
\item \textbf{A Heuristic Search library}: Arguably, the most complex part of
  the framework. This library would include a set of both uninformed and
  informed algorithms that have been adapted to the functional paradigm from
  their traditional implementations. Apart from the algorithms, the library
  will also provide the necessary building blocks for designing a new
  algorithm. These building blocks will be in fact used in the algorithms of
  the library, so the user can get a better intuition on how to use them just
  by checking the source code.
\item \textbf{The automation and benchmarking tools}: To offer a useful
  framework, it has to allow researchers to be able to perform batch
  experiments, as well as provide tools for benchmarking the code looking for
  possible performance improvements. That way, the framework will include
  several shortcuts for performing tests and comparisons between algorithm, as
  well as integration with benchmarking and optimization tools available for
  Haskell.
\item \textbf{Toy problems and other examples}: A small number of typical
  problems already implemented in Haskell, ready for the user to test or solve
  with the algorithms proposed (or developed).
\item \textbf{Tutorial, examples and documentation}: Due to the complexity and
  novelty of this framework, a comprehensive set of documentation has to be
  generated, and examples of use as well as several tutorials are planned to be
  written for the users to better understand how the library works
\end{enumerate}

\subsection{Analysis}

\subsubsection{Haskell's Foreign Function Interface}

There are also several remarks to be made before formally stating the
requirements. We could use two main approaches to design the library: use the
Haskell Foreign Function Interface or program everything in Haskell.\\

The Foreign Function Interface allows Haskell code to perform calls functions
written in other languages \cite{haskell98-ffi}. Using this interface, we could
program in a stateful and imperative way our search functions using for example
C/C++ and then call them from the library, that would serve just as a wrapper.
This would allow us to write high-performance code (also much more complicated
and low level than Haskell); performance that can be sometimes required in
this field.\\

However, using the FFI generates new dependencies for the project, that has to
be shipped in several languages. Also, this interface has to be carefully
treated and studied to prevent the creation of new corner cases that can induce
new bugs; and this interface is quite limited in the type of parameters that it
can receive, which is a great problem for this project. However, the most
important point is that, if the purpose of this library is to provide a way of
comparing functional search algorithms with some default ones (i.e
Breadth-First Search), comparing the user implementation with some code that is
indeed imperative and written in C/C++ seems pointless and misleading.\\

These reasons, and the fact that writing the whole library in Haskell is going
to be cleaner and more educative (as well as making it easier to provide some
construction blocks for the users) are the reasons why the FFI is omitted for
developing this library.\\

\subsubsection{Using Idiomatic Haskell}

Also, one of the main purposes of this library, is to explore the features of
the language and try to benefit from it as much as possible. Probably one of
the more interesting features is the non-strict evaluation, which could allow
us to introduce some interesting features.\\

In other frameworks \cite{cpp-search, cs4j, java-aima, hog2} a search returns a
node. This node, the solution, generally contains the state, the path recorded
to get there, and several run-time statistics like nodes expanded, open list
length, etc. If the search fails, usually the null element is returned. Since
this is not possible in Haskell, the intuitive approach is to use the
\texttt{Maybe} monad, that will let us return \texttt{Just <solution> |
  Nothing}. However, we can use a different structure that offers more
interesting possibilities: a list of solutions. This list can, indeed, be
infinite (or too big to be actually tractable), which already makes impossible
for other languages to manage it. However, due to the aforementioned laziness,
we can program the library to return this list without any problem: if the user
only asks for the first element (using for instance \texttt{head}), only this
element will be computed, performing the same computations as the traditional
libraries would do.\\

To collect the run-time statistics is not trivial as it can be in another
languages. Since we need to rely on recursion, collecting the total execution
statistics from the different stack calls requires a model called monad. %cite?
This monad will be explained later in the document, but to include it in the
library we need to perform modifications in the code to wrap the solutions in
it. If we try to wrap the list of solutions in the monad, we will be able to
obtain full run-time statistics, but no partial ones: if we ask only for the
head of the solutions it will be computed and nothing more, but if we want to
query the total nodes expanded the full search on the solution space has to be
performed for the monad to retrieve the number. Then, a list of solutions is
not the best structure if we want to use the monad.\\

The best solution is to offer the user two different flavors of the library:
one that implements a completely pure, fast and lazy search; and another one
that performs a ``more traditional'' search collecting run-time statistics.
This, however, will reduce code re-usability, since Haskell's strong typing
will not allow to share common code in some parts (for example, the data
structures code).
\newpage